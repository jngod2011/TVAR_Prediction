\documentclass[11pt]{beamer}
\usetheme{Copenhagen}
%%%%%%%%%%%%%%%%%%%%%%%%%%%%%%%%%%%%%%%%%%%%%%%%%%%%%%%%%%%
%%%%%  		 latin abbreviations                  %%%%%
%%%%%%%%%%%%%%%%%%%%%%%%%%%%%%%%%%%%%%%%%%%%%%%%%%%%%%%%%%%
\newcommand{\ie}{{\em i.e.,~}}
\newcommand{\eg}{{\em e.g.,~}}
\newcommand{\lcf}{{\em cf.~}}



%%%%%%%%%%%%%%%%%%%%%%%%%%%%%%%%%%%%%%%%%%%%%%%%%%%%%%%%%%%
%%%%%  		 Style math                         %%%%%
%%%%%%%%%%%%%%%%%%%%%%%%%%%%%%%%%%%%%%%%%%%%%%%%%%%%%%%%%%%

\newcommand{\nset}{\mathbb{N}}
\newcommand{\zset}{\mathbb{Z}}
\newcommand{\rset}{\mathbb{R}}
\newcommand{\cset}{\mathbb{C}}
\newcommand{\qset}{\mathbb{Q}}
\usepackage{xifthen}
\newcommand{\EE}[1][]{%
  \ifthenelse{\isempty{#1}}%
    {\mathbb{E}}% if #1 is empty
    {\mathbb{E}\left[#1\right]}% if #1 is not empty
}

\newcommand{\PP}[1][]{%
  \ifthenelse{\isempty{#1}}%
    {\mathbb{P}}% if #1 is empty
    {\mathbb{P}\left(#1\right)}% if #1 is not empty
}


\def\1{\mathbf 1}
\newcommand{\mb}{\mathbf}
\newcommand{\ud}{\,\mathrm{d}} 
\newcommand{\given}[1][{}]{\;\middle\vert\;{#1} }
\def\indep{\perp\!\!\!\perp}


%%%%%%%%%%%%%%%%%%%%%%%%%%%%%%%%%%%%%%%%%%%%%%%%%%%%%%%%%%%
%%%%%        defintion symbols                        %%%%%
%%%%%%%%%%%%%%%%%%%%%%%%%%%%%%%%%%%%%%%%%%%%%%%%%%%%%%%%%%%
\newcommand{\deq}{\stackrel{\rm def}{=}}


%%%%%%%%%%%%%%%%%%%%%%%%%%%%%%%%%%%%%%%%%%%%%%%%%%%%%%%%%%%
%%%%%        maths symbols                    	     %%%%%
%%%%%%%%%%%%%%%%%%%%%%%%%%%%%%%%%%%%%%%%%%%%%%%%%%%%%%%%%%%
\usepackage{amsopn}
\DeclareMathOperator{\sign}{sign}
\DeclareMathOperator{\Card}{Card}
\DeclareMathOperator{\rg}{rg}
\DeclareMathOperator{\tr}{tr}
\DeclareMathOperator{\corr}{corr}
\DeclareMathOperator{\var}{var}
\DeclareMathOperator{\vect}{vect}

\DeclareMathOperator{\cov}{cov}
\DeclareMathOperator{\pred}{pred}
\DeclareMathOperator{\Id}{Id}

\newcommand{\argmin}{\mathop{\mathrm{arg\,min}}}
\newcommand{\argmax}{\mathop{\mathrm{arg\,max}}}
\def\rang{\mathop{\rm rang}\nolimits}
\def\Ker{\mathop{\rm Ker}\nolimits}
\def\Im{\mathop{\rm Im}\nolimits}
\def\Vect{\mathop{\rm Vect}\nolimits}
\def\proj{\mathop{\rm proj}\nolimits}



\def\dim{\mathop{\rm dim}\nolimits}
\def\Var{\mathop{\rm Var}\nolimits}
\def\Cov{\mathop{\rm Cov}\nolimits}

\def\dom{\mathop{\rm dom}\nolimits}
\def\supp{\mathop{\rm supp}\nolimits}
\def\relint{\mathop{\rm relint}\nolimits}


\def\bfr{\mathbf{r}}
\def\bfx{\mathbf{x}}
\def\bfy{\mathbf{y}}
\def\bftheta{\boldsymbol{\theta}}


\def \cN{\mathcal{N}}
\def\calN{\cN}
\def\calL{\mathcal{L}}
\def\calM{\mathcal{M}}
\def\calH{\mathcal{H}}
\def \SSE{\text{SSE}}

\newcommand{\tnorm}[1]{{\left\vert\kern-0.25ex\left\vert\kern-0.25ex\left\vert #1 
    \right\vert\kern-0.25ex\right\vert\kern-0.25ex\right\vert}} % triple norm

\newcommand{\inner}[1]{{
\left\langle #1 \right\rangle}} % inner product

\newcommand{\norm}[1]{{
\left\| #1 \right\|}} % norm

\newcommand{\abs}[1]{{
\left| #1 \right|}} % abs

\newcommand{\piecewise}[1]{{
\left\lbrace \begin{array}{ll} #1 \end{array} \right. }} % for piecewise functions

\newcommand{\seg}[1]{{\left[ #1 \right]}} % closed segment [ ]
\newcommand{\osego}[1]{{\left] #1 \right[}} % open segment ] [
\newcommand{\oseg}[1]{{\left] #1 \right]}} % semi-open segment ] ]
\newcommand{\sego}[1]{{\left[ #1 \right[}} % semi-open segment [ [
\newcommand{\iseg}[1]{{\left\llbracket #1 \right\rrbracket}} % integer segment

\newcommand{\ens}[1]{{
\left\lbrace #1 \right\rbrace }} % ensemble

\newcommand{\floor}[1]{{
\left\lfloor #1 \right\rfloor }} % floor

%%%%%%%%%%%%%%%%%%%%%%%%%%%%%%%%%%%%%%%%%%%%%%%%%%%%%%%%%%%
%%%%% 	French, accents and font coding     	      %%%%%
%%%%%%%%%%%%%%%%%%%%%%%%%%%%%%%%%%%%%%%%%%%%%%%%%%%%%%%%%%%

\usepackage[french]{babel}
\usepackage{enumitem} 
\usepackage{fancyhdr}
\usepackage[utf8]{inputenc}
\usepackage{amsmath, amssymb, amsthm, amsfonts}			% for theorem definitions


\usepackage{graphicx}			% for images and graphics
\usepackage{stmaryrd}			% for \llbracket symbols [[ ]]

%%%%%%%%%%%%%%%%%%%%%%%%%%%%%%%%%%%%%%%%%%%%%%%%%%%%%%%%%%%
%%%%%  		 Theorems etc.                        %%%%%
%%%%%%%%%%%%%%%%%%%%%%%%%%%%%%%%%%%%%%%%%%%%%%%%%%%%%%%%%%%

\newtheorem{Thm}{Théorème}[section]
\newtheorem{Cor}{Corolaire}[Thm]
\newtheorem{Prop}{Proposition}[section]
\newtheorem{Lem}{Lemme}[section]
\theoremstyle{definition}
\newtheorem{Def}{Définition}[section]
\newtheorem{Ex}{Exemple}[section]
\newtheorem{Exo}{Exercice}[section]
\theoremstyle{remark}
\newtheorem*{Rque}{Remarque}
\theoremstyle{definition}
\newtheorem{Algo}{Algorithme}[section]


%%%%%%%%%%%%%%%%%%%%%%%%%%%%%%%%%%%%%%%%%%%%%%%%%%%%%%%%%
%%%%		For Algorithms
%%%%
%%%%%%%%%%%%%%%%%%%%%%%%%%%%%%%%%%%%%%%%%%%%%%%%%%%%%%%%%
\usepackage[ruled, vlined, french, algosection]{algorithm2e}
\SetKwBlock{KwInit}{Initialization :}{endInit}


%%%%%%%%%%%%%%%%%%%%%%%%%%%%%%%%%%%%%%%%%%%%%%%%%%%%%%%%%%%
%%%%%  		Code			 %%%%%
%%%%%%%%%%%%%%%%%%%%%%%%%%%%%%%%%%%%%%%%%%%%%%%%%%%%%%%%%%%
%\usepackage[section]{minted}
%\usemintedstyle{borland}
%\usepackage{color}
%\definecolor{bg}{rgb}{0.95,0.95,0.95}

\author{Thomas EBOLI et Amaury DURAND}
\title[SIGMA205]{SIGMA205 : Prédiction d'une série temporelle localement stationnaire}
\setbeamercovered{transparent} 
\setbeamertemplate{navigation symbols}{} 
\logo{\includegraphics[scale=0.5]{TelecomParisTech_logo}} 
\institute{Télécom ParisTech} 
\date{30 juin 2016} 
%\subject{} 

\AtBeginSubsection[]
{
  \begin{frame}
    \frametitle{Plan}
    \tableofcontents[
    currentsection, 
    currentsubsection, 
    ]
  \end{frame}
}
\begin{document}

\begin{frame}
\titlepage
\end{frame}

\begin{frame}
\frametitle{Plan}
\tableofcontents
\end{frame}

\section{Rappels sur les processus AR}
\subsection{Construction d'une solution stationnaire au second ordre}
\begin{frame}
\begin{block}{Equation AR}
\begin{equation}\label{eq:AR}
X_t = \sum_{i=1}^p \theta_i X_{t-i} + \epsilon_t
\tag{AR}
\end{equation}
$p\geq 1$, $(X_t)_{t\in \zset}$ un processus aléatoire à valeurs dans $\cset$, $\theta_1,\cdots,\theta_p \in \cset$ et $\epsilon \sim BB(0,\sigma^2)$
\end{block}

\begin{alertblock}{Théorème}
Soit $\Theta(z)=1-\sum_{k=1}^p \theta_k z^k$. Si $\forall |z|=1, \Theta(z)\neq 0$ \\
Alors il existe une solution stationnaire au second ordre
\end{alertblock}
\end{frame}

\subsection{Génération à partir de Levinson-Durbin}
\begin{frame}
\begin{algorithm}[H]
 \KwData{$(\kappa_1, \cdots , \kappa_d) \in \osego{-1,1}^d$}
 \KwResult{$(\theta_{i,p})_{p\in \iseg{1,d},i\in \iseg{1,p}}$ }
 \KwInit{ 
 	\For{$p = 1, \cdots, d$}{
 	$\theta_{p,p} = \kappa_p$\;
 	}
 }
\For{$p = 2, \cdots , d$}{
	\For{$m = 1, \cdots , p-1$}{
		$\theta_{m,p} = \theta_{m,p-1} - \kappa_p \theta_{p-m,p-1}$\;
	}
}
 \caption{Construction des $\theta$ à partir  des $\kappa$}
 \label{algo:construction}
\end{algorithm}
\end{frame}

\begin{frame}
\begin{alertblock}{Théorème}
\label{thm:kappa}
Soit $d \geq 1$. Alors quels que soient $(\kappa_1,\cdots, \kappa_d)\in \osego{-1,1}^d$, $\forall p \in \iseg{1,d}$, les coefficients $(\theta_{1,p}, \cdots, \theta_{p,p})$ construits par l'algorithme \ref{algo:construction} sont les coefficients d'un processus $AR(p)$ causal i.e 
$$
1-\sum_{k=1}^p \theta_{k,p} z^k \neq 0 \,\, \forall \abs{z} \leq 1
$$
\end{alertblock}
\end{frame}
\subsection{Densité spectrale de puissance}
\begin{frame}
\begin{Def}[DSP d'un AR(p)]
La DSP est définie pour $\lambda \in [-\pi, \pi[$ 
$
S_x(\lambda) = \frac{\sigma^2}{2\pi\abs{\Theta\left( e^{-i\lambda} \right)}^2}
$
où $\Theta(z) = 1 - \sum_{k=1}^p \theta_k z^k$ \\
\end{Def}
\begin{Prop}\label{prop:racines_phase}
Considérons $z_1,\cdots, z_p$ les inverses des racines de $\Theta(z)$ alors en notant $\forall n\in \iseg{1,p}, z_n = \rho_n e^{i \phi_n}$ avec $0 < \rho_n <1$ et $\phi_n \in [-\pi, \pi[$ on a, si $\forall n\in \iseg{1,p}, \rho_n$ est assez proche de $1$
$$
S_x(\lambda) \mbox{ admet un pic en $\lambda$ } \iff \exists n\in \iseg{1,p} \, \lambda = \phi_n
$$
\end{Prop}
\end{frame}

\section{Processus TVAR}
\subsection{Construction d'une solution stable}
\begin{frame}
\begin{block}{Equation TVAR et stabilité}
\textbf{Equation TVAR :}
\begin{equation} \label{eq:TVAR}
X_t = \sum_{i=1}^p \theta_i(t) X_{t-i} + \sigma(t) \epsilon_t
\tag{TVAR}
\end{equation}
où $\epsilon \overset{\rm{iid}}{\sim} BB(0,1)$. \\
\textbf{critère de stabilité :}
\begin{equation} \label{eq:stabilite}
\sup_{t\in \zset} \EE[|X_t|^2] < +\infty
\tag{S}
\end{equation}
\end{block}
\end{frame}

\begin{frame}
\textbf{Cas où p=1 : } $X_t = \theta(t)X_{t-1} + \sigma(t) \epsilon_t$
\begin{Prop}
Si $\sup_t |\theta(t)| < 1$ et $\sup_t |\sigma(t)| < +\infty$ alors il existe un unique processus $(X_t)_{t \in \zset}$ vérifiant à la fois \eqref{eq:TVAR} et la condition de stabilité \eqref{eq:stabilite}
\end{Prop}
Démonstration : On suppose \eqref{eq:TVAR} et \eqref{eq:stabilite}.
En itérant $k$ fois l'équation $\forall k \geq 0$
$$ 
X_t = \left( \prod_{j=0}^{k} \theta(t-j)  \right) X_{t-k-1} + \sum_{j=0}^k \left( \prod_{i=0}^{j-1} \theta(t-i) \right) \sigma(t-j)\epsilon_{t-j}
$$
But : $k \to +\infty$
\end{frame}
\begin{frame}
$M = \sup_t \EE[\abs{X_t}^2]$, $\theta_{\max} = \sup_t |\theta(t)|$ et $\sigma_{\max} = \sup_t |\sigma(t)|$
\[
\norm{\left( \prod_{j=0}^{k} \theta(t-j)  \right) X_{t-k-1} }_2^2 \leq \theta_{\max}^{2(k+1)} M \xrightarrow[k\to+\infty]{} 0
\]
De plus $\forall j \geq 0$
\[
\norm{\left( \prod_{i=0}^{j-1} \theta(t-i) \right) \sigma(t-j)\epsilon_{t-j}}_2 \leq \theta_{\max}^{j} \sigma_{\max} \Rightarrow \mbox{ série convergente }
\]
On fait tendre $k$ vers $+\infty$ :
$$
X_t = \sum_{j=0}^{+\infty} c_j  \epsilon_{t-j} \text{ avec } c_j = \left( \prod_{i=0}^{j-1} \theta(t-i) \right) \sigma(t-j)
$$
Réciproque : ok
\end{frame}
\begin{frame}
\textbf{Contre exemple pour p=2 :}
GRAPHIQUE A METTRE
\end{frame}
\begin{frame}
\textbf{Cas général :}
\begin{Def}[Nouvelle définition TVAR] \label{def:TVAR}
$\theta_1,\cdots, \theta_p$ et $\sigma$ définies sur $]-\infty,1]$ et $(\epsilon_t)_{t\in \zset} \overset{\rm{iid}}{\sim} BB(0,1)$. $T \geq 1$
\begin{equation}\label{eq:TVAR'}
\forall -\infty < t \leq T, X_{t,T} = \sum_{i=1}^p \theta_i\left( \frac{t}{T}\right) X_{t-i,T} + \sigma\left( \frac{t}{T} \right) \epsilon_t
\tag{TVAR'}
\end{equation}
\begin{equation}\label{eq:S'}
\sup_{-\infty < t\leq T} \EE[\abs{X_{t,T}}^2] < +\infty
\tag{S'}
\end{equation}
\end{Def}
\end{frame}
\begin{frame}
\begin{alertblock}{Théorème}
Hypothèses : 
\begin{itemize}
\item[$\bullet$] $\forall i\in \iseg{1,p}, \theta_i$ uniformément continue sur $]-\infty,1]$
\item[$\bullet$] $\sigma$ bornée sur $]-\infty,1]$. 
\item[$\bullet$] $\exists \delta \in ]0,1[, \Theta(z;u) \deq  1 - \sum_{i=1}^p \theta_i(u) z^i \neq 0 \, \forall \abs{z} < \delta^{-1}, u\in [0,1]$
\end{itemize}
Alors il existe $T_0 \geq 1$ tel que $\forall T \geq T_0$ il existe un unique processus $(X_{t,T})_{t\leq T}$ vérifiant \eqref{eq:TVAR'} et \eqref{eq:S'}.
\end{alertblock}
\end{frame}

\subsection{Génération}
\begin{frame}
\textbf{horizon de temps fini : } $t\in \iseg{0,T}$ \\
\textbf{Processus AR abstraits associés}
A $t\in \iseg{0,T}$ fixé $(\theta_1(t/T), \cdots , \theta_p(t/T))$ sont les coefficients d'un AR(p) causal 
\begin{alertblock}{Théorème}
$\forall \kappa_1, \cdots, \kappa_p$ continues de $[0,1]\to \osego{-1,1}$, les fonctions $\theta_{1,p},\cdots, \theta_{p,p}$ obtenues par l'algorithme \ref{algo:construction} à chaque instant $u$ sont les coefficients d'un processus TVAR(p) stable
\end{alertblock}
\end{frame}

\begin{frame}
\textbf{Implémentation TVAR(2) à partir des racines} GRAPHIQUE A METTRE

\end{frame}

\begin{frame}
\textbf{Implémentation TVAR(..) à partir des $\boldsymbol\kappa$} GRAPHIQUE A METTRE

\end{frame}

\subsection{Prédiction}
\begin{frame}
$\mb{X}_{t,T} = [X_{t,T}, X_{t-1,T}, \cdots , X_{t-p+1,T} ]^T$ \\
$\bftheta_{t,T} = \left[ \theta_1\left(\frac{t}{T}\right), \theta_2\left(\frac{t}{T}\right), \cdots, \theta_p\left(\frac{t}{T}\right) \right]^T$, 
$\sigma_{t,T} = \sigma\left(\frac{t}{T}\right)$ \\
\textbf{Critère : }$
\hat{\bftheta}_{t,T} = \argmin_{\bftheta \in \cset^p} \EE[ \abs{X_{t+1,T} - \bftheta^T \mb{X}_{t,T}}^2]
$
\begin{Def}[Estimateur NLMS de $\bftheta$]
On se donne un pas $\mu$, l'estimateur est le suivant : \\
$\hat{\bftheta}_{0,T}(\mu) = 0$ \\
$\hat{\bftheta}_{t+1,T}(\mu) = \hat{\bftheta}_{t,T}(\mu) + \mu ( X_{t+1,T} - \hat{\bftheta}_{t+1,T}(\mu)^T \mb{X}_{t,T} ) \frac{\mb{X}_{t,T}}{1+\mu \norm{\mb{X}_{t,T}}^2}
$
\end{Def}
\textbf{Prédicteur associé : }
$
\hat{X}_{t,T}(\mu) = \hat{\bftheta}_{t-1,T}(\mu)^T \mb{X}_{t-1,T}
$
\end{frame}
\begin{frame}
\textbf{Influence de $\mu$ : } GRAPHIQUE A METTRE
\end{frame}

\subsection{Agrégation des prédicteurs}
\begin{frame}
\begin{Def}[Agrégation]
$$
\forall t \in \iseg{1,T}, \hat{X}_{t,T} = \sum_{i=1}^N \alpha_{i,t} \hat{X}_{t,T}^{(i)}
$$
où $\alpha_t=(\alpha_{1,t},\cdots,\alpha_{N,t})\in S_N = \ens{\alpha \in \rset_+^N, \sum_{i=1}^N \alpha_{i} = 1}$ 
\end{Def}
\end{frame}

\begin{frame}
\textbf{Stratégie 1 : à partir du gradient de l'erreur quadratique}
$\forall i \in \iseg{1,N}, \forall t\in \iseg{1,T}$
$$
\hat{\alpha}_{i,t} = \frac{\exp\left( -2\eta \sum_{s=1}^{t-1} \left( \sum_{j=1}^N \hat{\alpha}_{j,s} \hat{X}_{s,T}^{(j)}-X_{s,T} \right)\hat{X}_{s,T}^{(i)}\right)}{\sum_{k=1}^N \exp\left( -2\eta \sum_{s=1}^{t-1} \left( \sum_{j=1}^N \hat{\alpha}_{j,s} \hat{X}_{s,T}^{(j)}-X_{s,T} \right)\hat{X}_{s,T}^{(k)}\right)}
$$
\textbf{Stratégie 2 : à partir de l'erreur quadratique}
$\forall i \in \iseg{1,N}, \forall t\in \iseg{1,T}$
$$
\hat{\alpha}_{i,t} = \frac{\exp\left( -\eta \sum_{s=1}^{t-1} \left( \hat{X}_{s,T}^{(i)} - X_{s,T}\right)^2\right)}{\sum_{k=1}^N \exp\left( -\eta \sum_{s=1}^{t-1} \left( \hat{X}_{s,T}^{(k)} - X_{s,T}\right)^2\right)}
$$
\end{frame}

\begin{frame}
\begin{alertblock}{Théorème (Performances)} $(X_{t,T})_{1\leq t \leq T}$ TVAR stable + régularité des prédicteurs : $\mathcal{E}(T) = \frac{1}{T} \sum_{t=1}^T \mathbb{E}[(\hat X_{t,T}-X_{t,T})^2]$
\begin{description}
\item{\underline{Stratégie 1 :}} Si $\sup_{t\in \zset} \EE[\abs{\epsilon_t}^4] < +\infty$ 
$ \displaystyle
\mathcal{E}(T) \leq \inf_{\alpha \in S_N} \frac{1}{T} \sum_{t=1}^T \mathbb{E}[(\hat X_{t,1}^{[\alpha]}-X_{t,T})^2] + \frac{\log(N)}{T \eta} + 2\eta C_4
$
\item{\underline{Stratégie 2 :}} Si il existe $p\geq 2$ tel que $\sup_{t\in \zset} \EE[\abs{\epsilon_t}^p] < +\infty$ 
$ \displaystyle
\mathcal{E}(T) \leq \min_{1 \leq i \leq N} \frac{1}{T} \sum_{t=1}^T \mathbb{E}[(\hat X_{t,T}^{(i)}-X_{t,T})^2] + \frac{\log(N)}{T \eta} + (2\eta)^{p/2-1}C_p
$
\end{description}
\end{alertblock}
\end{frame}

\section{Implémentations}
\subsection{Prédiction par agrégation}
\begin{frame}
A REMPLIR
\end{frame}
\subsection{Estimation DSP parole}
\begin{frame}
A REMPLIR
\end{frame}

\end{document}
